\section{Tallteori}
\begin{frame}{Divisjon og Modulær aritmetikk}
\begin{block}{Delelighet $a|b$ ($a$ deler $b$)}
\begin{itemize}
\item $a$ kan dele $b$ uten rest
\item $a|b$ er det samme som $\frac{b}{a}=c$ eller $b=a\cdot c$ med $c$ som heltall
\end{itemize}
\end{block}
\end{frame}

\begin{frame}
Integer representasjon
\end{frame}

\begin{frame}
Primes and greatest common divisors
\end{frame}

\begin{frame}
Solving Congruences
\end{frame}

% ==========================0

\section{Kryptografi}
\subsection*{Begrep}
\begin{frame}{Symmetrisk og asymmetrisk kryptografi}
\begin{block}{Symmetrisk kryptografi}
\begin{itemize}
\item Det finnes bare \textit{én} nøkkel, som begge personer bruker
\item Brukes for både kryptering og dekryptering
\item Eksempel: Caesar – $f(c) = (c+key) \% mod 26$
\end{itemize}
\end{block}
\begin{block}{Asymmetrisk kryptografi}
\begin{itemize}
\item Hver person har \textit{to} nøkler: Privat og offentlig
\item Kryptering med offentlig nøkkel av den andre personen
\item Dekryptering med privat nøkkel
\item Eksempel: RSA
\end{itemize}
\end{block}
\end{frame}

\subsection*{RSA}
\begin{frame}{RSA}
\begin{itemize}
\item Asymmetrisk kryptering med to nøkler for hver deltaker
\item Kryptering
	\begin{itemize}
	\item Offentlig nøkkel for kryptering
	\item Privat nøkkel for dekryptering
	\end{itemize}
\item Digitale sertifikater /signaturer
	\begin{itemize}
	\item Privat nøkkel for signering
	\item Offentlig nøkkel for verifisering
	\end{itemize}
\end{itemize}
\end{frame}


\begin{frame}
TODO: Fill with content @Lukas 
RSA
\end{frame}

\section{Stokastisitet}
\subsection{Counting}
\begin{frame}
TODO: Fill with content @Lukas 
Basics of Counting
Pigeonhole Principle
Permutations and Combinations
Binomial conefficients and identities
\end{frame}

\subsection{Sannsynligheter}
\begin{frame}
TODO: Fill with content @Lukas 
Bayes theorem, probabilitiy theory
discrete probability
expected value and variance
\end{frame}


